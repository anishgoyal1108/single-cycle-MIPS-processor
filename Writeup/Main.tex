% Options for packages loaded elsewhere
% Options for packages loaded elsewhere
\PassOptionsToPackage{unicode}{hyperref}
\PassOptionsToPackage{hyphens}{url}
\PassOptionsToPackage{dvipsnames,svgnames,x11names}{xcolor}
%
\documentclass[
  letterpaper,
  DIV=11,
  numbers=noendperiod]{scrartcl}
\usepackage{xcolor}
\usepackage{amsmath,amssymb}
\setcounter{secnumdepth}{5}
\usepackage{iftex}
\ifPDFTeX
  \usepackage[T1]{fontenc}
  \usepackage[utf8]{inputenc}
  \usepackage{textcomp} % provide euro and other symbols
\else % if luatex or xetex
  \usepackage{unicode-math} % this also loads fontspec
  \defaultfontfeatures{Scale=MatchLowercase}
  \defaultfontfeatures[\rmfamily]{Ligatures=TeX,Scale=1}
\fi
\usepackage{lmodern}
\ifPDFTeX\else
  % xetex/luatex font selection
\fi
% Use upquote if available, for straight quotes in verbatim environments
\IfFileExists{upquote.sty}{\usepackage{upquote}}{}
\IfFileExists{microtype.sty}{% use microtype if available
  \usepackage[]{microtype}
  \UseMicrotypeSet[protrusion]{basicmath} % disable protrusion for tt fonts
}{}
\makeatletter
\@ifundefined{KOMAClassName}{% if non-KOMA class
  \IfFileExists{parskip.sty}{%
    \usepackage{parskip}
  }{% else
    \setlength{\parindent}{0pt}
    \setlength{\parskip}{6pt plus 2pt minus 1pt}}
}{% if KOMA class
  \KOMAoptions{parskip=half}}
\makeatother
% Make \paragraph and \subparagraph free-standing
\makeatletter
\ifx\paragraph\undefined\else
  \let\oldparagraph\paragraph
  \renewcommand{\paragraph}{
    \@ifstar
      \xxxParagraphStar
      \xxxParagraphNoStar
  }
  \newcommand{\xxxParagraphStar}[1]{\oldparagraph*{#1}\mbox{}}
  \newcommand{\xxxParagraphNoStar}[1]{\oldparagraph{#1}\mbox{}}
\fi
\ifx\subparagraph\undefined\else
  \let\oldsubparagraph\subparagraph
  \renewcommand{\subparagraph}{
    \@ifstar
      \xxxSubParagraphStar
      \xxxSubParagraphNoStar
  }
  \newcommand{\xxxSubParagraphStar}[1]{\oldsubparagraph*{#1}\mbox{}}
  \newcommand{\xxxSubParagraphNoStar}[1]{\oldsubparagraph{#1}\mbox{}}
\fi
\makeatother


\usepackage{longtable,booktabs,array}
\usepackage{calc} % for calculating minipage widths
% Correct order of tables after \paragraph or \subparagraph
\usepackage{etoolbox}
\makeatletter
\patchcmd\longtable{\par}{\if@noskipsec\mbox{}\fi\par}{}{}
\makeatother
% Allow footnotes in longtable head/foot
\IfFileExists{footnotehyper.sty}{\usepackage{footnotehyper}}{\usepackage{footnote}}
\makesavenoteenv{longtable}
\usepackage{graphicx}
\makeatletter
\newsavebox\pandoc@box
\newcommand*\pandocbounded[1]{% scales image to fit in text height/width
  \sbox\pandoc@box{#1}%
  \Gscale@div\@tempa{\textheight}{\dimexpr\ht\pandoc@box+\dp\pandoc@box\relax}%
  \Gscale@div\@tempb{\linewidth}{\wd\pandoc@box}%
  \ifdim\@tempb\p@<\@tempa\p@\let\@tempa\@tempb\fi% select the smaller of both
  \ifdim\@tempa\p@<\p@\scalebox{\@tempa}{\usebox\pandoc@box}%
  \else\usebox{\pandoc@box}%
  \fi%
}
% Set default figure placement to htbp
\def\fps@figure{htbp}
\makeatother


% definitions for citeproc citations
\NewDocumentCommand\citeproctext{}{}
\NewDocumentCommand\citeproc{mm}{%
  \begingroup\def\citeproctext{#2}\cite{#1}\endgroup}
\makeatletter
 % allow citations to break across lines
 \let\@cite@ofmt\@firstofone
 % avoid brackets around text for \cite:
 \def\@biblabel#1{}
 \def\@cite#1#2{{#1\if@tempswa , #2\fi}}
\makeatother
\newlength{\cslhangindent}
\setlength{\cslhangindent}{1.5em}
\newlength{\csllabelwidth}
\setlength{\csllabelwidth}{3em}
\newenvironment{CSLReferences}[2] % #1 hanging-indent, #2 entry-spacing
 {\begin{list}{}{%
  \setlength{\itemindent}{0pt}
  \setlength{\leftmargin}{0pt}
  \setlength{\parsep}{0pt}
  % turn on hanging indent if param 1 is 1
  \ifodd #1
   \setlength{\leftmargin}{\cslhangindent}
   \setlength{\itemindent}{-1\cslhangindent}
  \fi
  % set entry spacing
  \setlength{\itemsep}{#2\baselineskip}}}
 {\end{list}}
\usepackage{calc}
\newcommand{\CSLBlock}[1]{\hfill\break\parbox[t]{\linewidth}{\strut\ignorespaces#1\strut}}
\newcommand{\CSLLeftMargin}[1]{\parbox[t]{\csllabelwidth}{\strut#1\strut}}
\newcommand{\CSLRightInline}[1]{\parbox[t]{\linewidth - \csllabelwidth}{\strut#1\strut}}
\newcommand{\CSLIndent}[1]{\hspace{\cslhangindent}#1}



\setlength{\emergencystretch}{3em} % prevent overfull lines

\providecommand{\tightlist}{%
  \setlength{\itemsep}{0pt}\setlength{\parskip}{0pt}}



 


\addtokomafont{disposition}{\rmfamily}
\usepackage{amsmath, xparse}
\usepackage{fancyvrb, fvextra}
\usepackage{listings}
\usepackage{svg}
\usepackage{amsfonts}
\usepackage{amssymb}
\usepackage{graphicx}
\usepackage{tcolorbox}
\usepackage{multicol}
\usepackage{physics}
\usepackage{listings}
\usepackage{tikz}
\usetikzlibrary{arrows, positioning, calc, intersections}
\usetikzlibrary{decorations.pathreplacing, decorations.markings}
\usepackage{pgfplots}
\usepackage{tikz-3dplot}
\usepackage[table,x11names]{xcolor}
\usepackage{systeme}
\usepackage{xifthen}
\usepackage{karnaugh-map}
\usepackage{bm}
\usepackage{caption}
\definecolor{cornflower}{rgb}{0.12549, 0.29020, 0.52941}
\DefineVerbatimEnvironment{Highlighting}{Verbatim}{breaklines,breakanywhere,commandchars=\\\{\}}
\renewcommand{\lstlistlistingname}{List of Listings}
\lstset{
  basicstyle=\ttfamily\small,
  breaklines=true,
  breakatwhitespace=false,
  columns=flexible,
  keepspaces=true,
  showstringspaces=false,
  commentstyle=\color{green!50!black},
  keywordstyle=\color{blue},
  stringstyle=\color{red},
  numberstyle=\tiny\color{gray},
  frame=single,
  captionpos=b,
}
\KOMAoption{captions}{tableheading}
\makeatletter
\@ifpackageloaded{caption}{}{\usepackage{caption}}
\AtBeginDocument{%
\ifdefined\contentsname
  \renewcommand*\contentsname{Table of contents}
\else
  \newcommand\contentsname{Table of contents}
\fi
\ifdefined\listfigurename
  \renewcommand*\listfigurename{List of Figures}
\else
  \newcommand\listfigurename{List of Figures}
\fi
\ifdefined\listtablename
  \renewcommand*\listtablename{List of Tables}
\else
  \newcommand\listtablename{List of Tables}
\fi
\ifdefined\figurename
  \renewcommand*\figurename{Figure}
\else
  \newcommand\figurename{Figure}
\fi
\ifdefined\tablename
  \renewcommand*\tablename{Table}
\else
  \newcommand\tablename{Table}
\fi
}
\@ifpackageloaded{float}{}{\usepackage{float}}
\floatstyle{ruled}
\@ifundefined{c@chapter}{\newfloat{codelisting}{h}{lop}}{\newfloat{codelisting}{h}{lop}[chapter]}
\floatname{codelisting}{Listing}
\newcommand*\listoflistings{\listof{codelisting}{List of Listings}}
\makeatother
\makeatletter
\makeatother
\makeatletter
\@ifpackageloaded{caption}{}{\usepackage{caption}}
\@ifpackageloaded{subcaption}{}{\usepackage{subcaption}}
\makeatother
\usepackage{bookmark}
\IfFileExists{xurl.sty}{\usepackage{xurl}}{} % add URL line breaks if available
\urlstyle{same}
\hypersetup{
  colorlinks=true,
  linkcolor={blue},
  filecolor={Maroon},
  citecolor={Blue},
  urlcolor={Blue},
  pdfcreator={LaTeX via pandoc}}


\author{}
\date{}
\begin{document}

\begin{titlepage}
  \vspace{5in}
    \newcommand{\HRule}{\rule{\linewidth}{0.5mm}}
    \center
    \textsc{\LARGE Georgia Southern University \\ \vspace{0.2cm} \Large Allen E. Paulson College of Engineering and Computing \\ Department of Electrical and Computer Engineering}\\[0.3cm]

    \HRule \\[0.4cm]
    { \huge \bfseries Computer Systems Design Final Project}\\[0.20cm]
    { \Large \bfseries Single-Cycle MIPS Processor}\\[0.10cm]
    \HRule \\[0.6cm]
    
    \begin{minipage}{0.4\textwidth}
    \begin{flushleft} \Large
    Anish Goyal
    \end{flushleft}
    \end{minipage}
    ~
    \begin{minipage}{0.4\textwidth}
    \begin{flushright} \Large
    Dr. Mohammad Ahad \\ Associate Professor 
    \end{flushright}
    \end{minipage}\\[0.5cm]
    
    {\huge December 5, 2025}\\[0.5cm]

    \vspace{0.1cm}

    \includegraphics[width=0.60\textwidth]{img/logo.png}\\

    \begin{abstract}
      \section*{Abstract}
      \noindent I designed and implemented a single-cycle MIPS processor using VHDL and Intel Quartus Prime. The processor supports R-type instructions (add, sub) and I-type instructions (lw, sw). I constructed the datapath using schematic capture in Quartus, connecting individual VHDL components including the program counter, instruction memory, register file, ALU, ALU controller, main control unit, data memory, multiplexers, and sign extension unit. I validated the processor through functional simulation in Questa/ModelSim, verifying correct instruction execution across four test instructions. The simulation results confirm proper ALU operations, register reads and writes, and memory access behavior.
    \end{abstract}
  \end{titlepage}

\newpage

\renewcommand*\contentsname{Table of Contents}
{
\hypersetup{linkcolor=}
\setcounter{tocdepth}{4}
\tableofcontents
}
\listoffigures
\listoftables

\clearpage
\lstlistoflistings
\clearpage

\clearpage

\section{Introduction}\label{introduction}

The MIPS (Microprocessor without Interlocked Pipeline Stages)
architecture represents one of the most influential reduced instruction
set computing (RISC) designs in computer engineering history. I
undertook this project to gain practical experience implementing a
processor datapath and control unit in hardware description language.
The single-cycle processor architecture executes each instruction in
exactly one clock cycle, making it an ideal starting point for
understanding how hardware components work together to fetch, decode,
and execute machine instructions.

In this project, I implemented a single-cycle MIPS processor using VHDL
and Intel Quartus Prime. The processor supports a subset of the MIPS
instruction set, including R-type arithmetic operations (add, sub) and
I-type memory operations (lw, sw). I based my design on the classic MIPS
datapath described by Patterson and Hennessy in their textbook {[}1{]}.
Figure~\ref{fig-textbook-datapath} shows the reference datapath that
guided my implementation.

\begin{figure}

\centering{

\pandocbounded{\includegraphics[keepaspectratio]{img/full-data-path-from-textbook.png}}

}

\caption{\label{fig-textbook-datapath}Single-cycle MIPS datapath from
Patterson and Hennessy textbook}

\end{figure}%

The datapath consists of several key components that work together
during instruction execution. The program counter (PC) holds the address
of the current instruction. The instruction memory stores the program
and outputs the 32-bit instruction at the PC address. The control unit
decodes the instruction opcode and generates control signals. The
register file provides two read ports and one write port for accessing
the 32 general-purpose registers. The ALU performs arithmetic and logic
operations. The data memory handles load and store operations.
Multiplexers route data between components based on the instruction
type.

My goal was to implement each of these components as individual VHDL
modules, then connect them using the Quartus schematic editor to create
a complete working processor. I validated the design through functional
simulation, testing a sequence of four instructions that exercise both
R-type and I-type operations.

\clearpage

\section{Equipment}\label{equipment}

I used the following tools and equipment for this project. Intel Quartus
Prime Lite Edition version 23.1.1 served as the primary development
environment for writing VHDL code and creating the block diagram
schematic. I used Questa (ModelSim) for functional simulation and
waveform analysis. The target platform was the DE10-Standard development
board featuring the Cyclone V 5CSXFC6D6F31C6 FPGA, though I focused on
simulation rather than hardware deployment for this project.

\clearpage

\section{Methods}\label{methods}

I followed a modular design approach based on the EENG 5342 course
materials {[}2{]}. I began by identifying all the components required
for the single-cycle datapath. I then implemented each component as a
separate VHDL entity with its own architecture. This modular approach
allowed me to test individual components before integration.

For each component, I defined the input and output ports based on the
datapath diagram. I wrote the behavioral VHDL code to implement the
required functionality. I compiled each module in Quartus to verify
correct syntax and generate the schematic symbol.

After completing all components, I created a new Block Diagram File
(.bdf) in Quartus. I instantiated each VHDL symbol and connected them
according to the datapath architecture. I paid careful attention to bus
widths and signal naming conventions to ensure proper connectivity.

For testing, I initialized the instruction memory with four test
instructions. I initialized the register file with values 0 through 31
in registers \$0 through \$31. I ran functional simulation in Questa
with a 1 microsecond clock period. I examined the waveforms to verify
correct operation of each instruction.

\clearpage

\section{Results}\label{results}

I successfully implemented all components of the single-cycle MIPS
processor. This section describes each component, its role in the
datapath, and the corresponding VHDL implementation.

\subsection{Program Counter}\label{program-counter}

The program counter serves as the fundamental sequencing element of the
processor. It maintains the address of the instruction currently being
executed and determines the flow of program execution.
Figure~\ref{fig-pc} shows the PC component as implemented in Quartus.
The complete VHDL implementation appears in Listing \ref{lst-pc} in the
Appendix.

\begin{figure}

\centering{

\pandocbounded{\includegraphics[keepaspectratio]{img/pc.png}}

}

\caption{\label{fig-pc}Program Counter component in Quartus}

\end{figure}%

The PC has a 32-bit input port (PCin) that receives the next address
value from the PC multiplexer, and a 32-bit output port (PCout) that
provides the current address to the instruction memory. The update
occurs synchronously on the rising edge of the clock signal, ensuring
that address changes happen at predictable times aligned with the
processor clock.

The PC implements a simple D flip-flop behavior at the 32-bit level.
When the clock transitions from low to high, the PC latches the value
present on PCin and immediately propagates it to PCout. This synchronous
operation ensures that all downstream components receive a stable
address throughout the clock cycle. The PC has no reset signal in my
implementation, meaning it retains whatever value is present at
power-on. For a production processor, a reset capability would be
essential to ensure deterministic startup behavior.

In the single-cycle architecture, the PC updates exactly once per
instruction. The new PC value comes from one of two sources selected by
a multiplexer: PC+4 for sequential execution, or a branch target address
for taken branches. My implementation only supports sequential execution
since I did not implement branch instructions in the final design,
though the datapath includes the necessary branch calculation hardware.

\clearpage

\subsection{PC Adder}\label{pc-adder}

The PC adder is a dedicated 32-bit adder that computes the address of
the next sequential instruction. Figure~\ref{fig-pc-adder} shows this
component as it appears in the Quartus schematic. The VHDL
implementation uses a simple combinational addition operation as shown
in Listing \ref{lst-adder}.

\begin{figure}

\centering{

\pandocbounded{\includegraphics[keepaspectratio]{img/pc_adder.png}}

}

\caption{\label{fig-pc-adder}PC Adder component in Quartus}

\end{figure}%

In the MIPS architecture, all instructions are exactly 32 bits (4 bytes)
wide and must be word-aligned in memory. This means instruction
addresses are always multiples of 4. The PC adder takes the current PC
value as one input and the constant value 4 as the other input,
producing PC+4 as output. This output represents the address of the next
sequential instruction in memory.

The adder operates combinationally, meaning the output changes
immediately whenever the input changes, with only the propagation delay
of the adder logic. I implemented the adder using the built-in addition
operator in VHDL, which synthesizes to a ripple-carry adder structure.
For a 32-bit addition, this introduces some propagation delay as the
carry ripples through all 32 bit positions. In a higher-performance
processor, a carry-lookahead or carry-select adder might be used to
reduce this critical path delay.

The PC+4 value feeds into a multiplexer that selects between sequential
execution and branch targets. In my implementation, since I did not
fully implement branch logic, the PC+4 value always becomes the next PC
value, creating simple sequential program flow.

\clearpage

\subsection{Instruction Memory}\label{instruction-memory}

The instruction memory (IMEM) stores the program code and provides read
access to instructions based on the address from the PC.
Figure~\ref{fig-im} shows the instruction memory component in the
Quartus schematic. The complete VHDL implementation appears in Listing
\ref{lst-im}.

\begin{figure}

\centering{

\pandocbounded{\includegraphics[keepaspectratio]{img/im.png}}

}

\caption{\label{fig-im}Instruction Memory component in Quartus}

\end{figure}%

I implemented the instruction memory as a read-only memory (ROM) using a
VHDL array type. The memory contains 4 words, each 32 bits wide, holding
the test program. The memory operates combinationally rather than
synchronously. This means the output instruction updates immediately
when the address changes, with only combinational propagation delay.
This is critical for single-cycle operation because the instruction must
be available within the same clock cycle that the PC is updated.

The address input is a full 32-bit value, but since each instruction
occupies 4 bytes, I divide the address by 4 to obtain the word index
into the array. In VHDL, I use the expression
\texttt{to\_integer(unsigned(InstAddress)\ /\ 4)} to perform this
conversion. This means address 0 accesses instruction 0, address 4
accesses instruction 1, address 8 accesses instruction 2, and so on.
This matches the MIPS convention where instructions are word-aligned.

I preloaded four specific test instructions into the memory during
synthesis: 0x012A4020 (add \$8, \$9, \$10), 0x01285022 (sub \$10, \$9,
\$8), 0x8E510000 (lw \$17, 0(\$18)), and 0xAE290004 (sw \$9, 4(\$17)).
These instructions test both R-type arithmetic operations and I-type
memory operations. In a real processor, the instruction memory would be
much larger and would typically be loaded from external storage at boot
time. For FPGA implementation, the memory could be initialized from a
.mif file, allowing program changes without resynthesizing the design.

\clearpage

\subsection{Main Control Unit}\label{main-control-unit}

The main control unit serves as the brain of the processor, decoding
instructions and orchestrating the operation of all datapath components.
It examines the 6-bit opcode field (bits 31-26) of each instruction and
generates the appropriate control signals. Figure~\ref{fig-main-control}
shows the main control unit in the Quartus schematic. The VHDL
implementation appears in Listing \ref{lst-maincontrol}.

\begin{figure}

\centering{

\pandocbounded{\includegraphics[keepaspectratio]{img/main_control.png}}

}

\caption{\label{fig-main-control}Main Control Unit component in Quartus}

\end{figure}%

The control unit outputs eight control signals that govern datapath
behavior. RegDst selects which instruction field specifies the
destination register: for R-type instructions, this is the rd field
(bits 15-11), while for I-type instructions, it is the rt field (bits
20-16). ALUSrc selects the second operand for the ALU: either the second
register value (for R-type) or the sign-extended immediate (for I-type).
MemtoReg selects the data to write back to the register file: either the
ALU result (for arithmetic instructions) or memory read data (for load
instructions). RegWrite enables writing to the register file at the end
of the instruction cycle. MemRead and MemWrite enable reading from and
writing to data memory, respectively. Branch combines with the ALU Zero
flag to enable branch execution. ALUOp provides a 2-bit code that tells
the ALU controller how to interpret the instruction.

For R-type instructions (opcode 000000), the control unit asserts RegDst
to select the rd field and sets RegWrite to enable writing the ALU
result. ALUOp is set to 10, indicating that the ALU controller should
examine the function field to determine the specific operation. The
remaining control signals are deasserted since R-type instructions do
not access memory or perform branches.

For load word instructions (opcode 100011), the control unit asserts
ALUSrc to route the sign-extended immediate to the ALU for address
calculation. It sets ALUOp to 00, which tells the ALU controller to
perform addition regardless of the function field. MemRead is asserted
to read from data memory, and MemtoReg is set to route the memory data
(rather than ALU result) to the register file. RegWrite is asserted with
a 10ns delay to ensure the memory read completes before the write
occurs.

For store word instructions (opcode 101011), the configuration is
similar to load word, but MemWrite is asserted instead of MemRead, and
RegWrite is deasserted since store instructions do not write to
registers. The ALU still performs address calculation by adding the base
register and immediate offset.

For branch equal instructions (opcode 000100), ALUOp is set to 01 to
force a subtraction operation, which sets the Zero flag when the two
register values are equal. The Branch signal is asserted to enable the
branch address multiplexer. However, I did not fully implement the
branch logic in my final processor, so branch instructions do not
execute correctly.

The control unit operates purely combinationally, generating all control
signals based solely on the opcode. This allows control signals to be
available within the same clock cycle as the instruction fetch, which is
essential for single-cycle operation. The trade-off is that every
instruction takes a full clock cycle even if it could complete faster,
and the clock period must be long enough to accommodate the slowest
possible instruction path.

\clearpage

\subsection{Register File}\label{register-file}

The register file forms the primary storage for operands and results
during computation. It implements 32 general-purpose registers following
the MIPS register architecture, with each register holding 32 bits. The
register file provides two independent read ports and one write port,
allowing a single instruction to read two source operands and write one
result simultaneously. Figure~\ref{fig-register-file} shows the register
file component in the Quartus schematic. The VHDL implementation appears
in Listing \ref{lst-registerfile}.

\begin{figure}

\centering{

\pandocbounded{\includegraphics[keepaspectratio]{img/register_file.png}}

}

\caption{\label{fig-register-file}Register File component in Quartus}

\end{figure}%

The register file receives five inputs: three 5-bit register addresses
and two data signals. RegOne and RegTwo specify the registers to read
from ports 1 and 2, respectively. RegThree specifies the destination
register for write operations. DataIn carries the 32-bit value to write.
RegWrite serves as a write enable signal. The register file produces two
32-bit outputs: RegOutOne and RegOutTwo, corresponding to the values
stored in the registers addressed by RegOne and RegTwo.

I implemented the register file using a VHDL array type containing 32
elements of 32-bit vectors. The read operations are purely
combinational. When RegOne or RegTwo changes, the corresponding output
updates immediately through a continuous assignment statement. This is
implemented using the VHDL expressions
\texttt{RegOutOne\ \textless{}=\ myarray(TO\_INTEGER(UNSIGNED(RegOne)))}
and
\texttt{RegOutTwo\ \textless{}=\ myarray(TO\_INTEGER(UNSIGNED(RegTwo)))}.
These statements continuously drive the outputs with the current
register values.

The write operation is controlled by the RegWrite signal. I implemented
this using a process that is sensitive to RegWrite. When RegWrite
transitions to high, the process executes and writes DataIn to the
register specified by RegThree. In a real MIPS processor, register \$0
is hardwired to zero and cannot be written. My implementation does not
enforce this constraint, meaning writes to register 0 would overwrite
the zero value. A production implementation would add logic to ignore
writes when RegThree is zero.

For testing and debugging, I initialized all registers with their index
values at synthesis time. This means register \$0 contains 0, register
\$1 contains 1, register \$2 contains 2, and so forth up to register
\$31 containing 31. This initialization pattern makes it easy to verify
correct operation during simulation because the register contents are
predictable. I can see whether the processor is reading from the correct
registers by comparing the outputs to the expected index values.

The register file operates on an unusual clock edge compared to typical
synchronous designs. Rather than using the main processor clock
directly, the write enable signal RegWrite serves as the synchronization
point. This works for single-cycle operation but would need modification
for a pipelined design where writes must align with specific pipeline
stages.

\clearpage

\subsection{Sign Extension Unit}\label{sign-extension-unit}

The sign extension unit converts the 16-bit immediate field from I-type
instructions into 32-bit signed values compatible with the rest of the
32-bit datapath. This conversion must preserve the arithmetic sign of
the original value. Figure~\ref{fig-sx} shows the sign extension
component in the Quartus schematic. The VHDL implementation appears in
Listing \ref{lst-signext}.

\begin{figure}

\centering{

\pandocbounded{\includegraphics[keepaspectratio]{img/sx.png}}

}

\caption{\label{fig-sx}Sign Extension Unit component in Quartus}

\end{figure}%

The unit examines bit 15 of the 16-bit input, which represents the sign
bit in two's complement representation. If bit 15 is 0, the value is
positive, and the unit creates a 32-bit output by concatenating 16 zeros
with the original 16-bit value. If bit 15 is 1, the value is negative,
and the unit concatenates 16 ones with the original 16-bit value. This
process preserves the two's complement signed value across the width
change.

For example, the 16-bit value 0x0004 (decimal 4) has bit 15 equal to 0.
Sign extension produces 0x00000004, which correctly represents positive
4 in 32 bits. The 16-bit value 0xFFFC (decimal -4 in two's complement)
has bit 15 equal to 1. Sign extension produces 0xFFFFFFFC, which
correctly represents negative 4 in 32-bit two's complement.

The sign extension unit operates purely combinationally with no clock or
state. The output updates immediately when the input changes. This is
necessary because the extended immediate value must be available within
the same cycle for use by the ALU in address calculations or arithmetic
operations.

I implemented the logic using a VHDL process with a conditional
statement that checks the sign bit and concatenates the appropriate
prefix. The VHDL concatenation operator (\&) combines the 16-bit prefix
with the original value to form the 32-bit result.

\clearpage

\subsection{ALU}\label{alu}

The arithmetic logic unit (ALU) performs the computational heart of
instruction execution. It implements a variety of arithmetic and logical
operations required by MIPS instructions. Figure~\ref{fig-alu} shows the
ALU component in the Quartus schematic. The complete VHDL implementation
appears in Listing \ref{lst-alu}.

\begin{figure}

\centering{

\pandocbounded{\includegraphics[keepaspectratio]{img/alu.png}}

}

\caption{\label{fig-alu}ALU component in Quartus}

\end{figure}%

The ALU receives two 32-bit operands labeled A and B, along with a 4-bit
operation code (OPCODE) that selects the operation to perform. It
produces two outputs: a 32-bit result (R) containing the computation
result, and a single-bit Zero flag that indicates whether the result
equals zero. The Zero flag is particularly important for conditional
branch instructions, which branch when two values are equal (i.e., their
difference is zero).

I implemented the ALU using a large VHDL process with a cascading
if-elsif structure that decodes the OPCODE and performs the
corresponding operation. The supported operations include: NOT A (0000),
NOT B (0001), addition (0010), NAND (0011), OR (0100), NOR (0101),
subtraction (0110), XNOR (0111), AND (1000), set-less-than (1001),
increment A (1010), decrement A (1011), increment B (1100), decrement B
(1101), negate A (1110), and negate B (1111). Not all of these
operations are used by the MIPS instructions I implemented, but they
were included in the ALU design from the class reference material.

For the addition operation (OPCODE 0010), the ALU computes R = A + B
using VHDL's built-in addition operator. This operator works on
std\_logic\_vector types and performs unsigned binary addition with
carry propagation. The result wraps around on overflow without any
indication. For the subtraction operation (OPCODE 0110), the ALU
computes R = A - B. I implemented this using VHDL's subtraction
operator. The subtraction also uses an internal signal Rtemp to hold the
result before assigning it to R. This allows the conditional logic to
examine the result and set the Zero flag to `1' when Rtemp equals
0x00000000, otherwise Zero is set to `0'.

The logical operations (AND, OR, NOR, NAND) apply bitwise operations to
corresponding bits of A and B. For example, AND (1000) produces a result
where bit i is 1 only when both bit i of A and bit i of B are 1. The
set-less-than operation (1001) performs a signed comparison. If A is
less than B (treating both as signed two's complement numbers), the ALU
outputs 0x00000001. Otherwise it outputs 0x00000000. This implements the
MIPS slt instruction.

One notable issue I encountered during development involved opcode
conflicts. The original class code used 0110 for XOR, but the MIPS
subtract operation also uses 0110. I resolved this by removing the XOR
case and ensuring subtraction worked correctly. This highlights the
importance of maintaining consistent opcode mappings between the control
unit, ALU controller, and ALU itself.

The ALU operates purely combinationally. When the inputs or opcode
change, the output updates after the propagation delay through the
combinational logic. For a 32-bit operation, this delay can be
substantial, especially for operations like addition and subtraction
that have carry chains. The ALU propagation delay forms part of the
critical path that determines the minimum clock period for the
processor.

\clearpage

\subsection{ALU Controller}\label{alu-controller}

The ALU controller serves as an intermediary between the main control
unit and the ALU, translating high-level instruction types into specific
ALU operation codes. It combines the 2-bit ALUOp signal from the main
control unit with the 6-bit function field (bits 5-0) from R-type
instructions to generate the final 4-bit ALU control signal.
Figure~\ref{fig-alu-control} shows the ALU controller component in the
Quartus schematic. The VHDL implementation appears in Listing
\ref{lst-alucontroller}.

\begin{figure}

\centering{

\pandocbounded{\includegraphics[keepaspectratio]{img/alu_control.png}}

}

\caption{\label{fig-alu-control}ALU Controller component in Quartus}

\end{figure}%

The ALU controller receives two inputs: ALUOp (2 bits) from the main
control unit and Instr (6 bits) representing the function field of the
current instruction. It produces a 4-bit output Y that directly controls
the ALU operation. The controller implements a two-level decoding scheme
that simplifies the main control unit design.

When ALUOp is 00, the controller outputs 0010 (add) regardless of the
function field. The main control unit sets ALUOp to 00 for load and
store instructions, which need to compute memory addresses by adding a
base register and an offset. Since address calculation always requires
addition, the function field is ignored.

When ALUOp is 01, the controller outputs 0110 (subtract) regardless of
the function field. The main control unit sets ALUOp to 01 for branch
equal instructions. To test whether two values are equal, the processor
subtracts them and checks if the result is zero. Again, the function
field is not meaningful for this instruction type.

When ALUOp is 10, the controller examines the function field to
determine the specific R-type operation. Function 100000 (decimal 32)
corresponds to add, producing ALU control 0010. Function 100010 (decimal
34) corresponds to subtract, producing 0110. Function 100100 (decimal
36) corresponds to AND, producing 0000. Function 100101 (decimal 37)
corresponds to OR, producing 0001. Function 101010 (decimal 42)
corresponds to set-less-than, producing 0111. Note that the ALU control
codes were derived from the ALU implementation and do not follow any
standard encoding scheme.

This two-level decoding approach separates concerns: the main control
unit determines the general category of operation based on the opcode,
and the ALU controller refines this into a specific operation based on
the function field. For I-type instructions, only the opcode matters.
For R-type instructions, both opcode and function field matter.
Table~\ref{tbl-opcode} summarizes the complete mapping from instruction
fields to ALU control signals.

The ALU controller operates combinationally using nested if-elsif
statements in VHDL. The outer level checks ALUOp, and for the R-type
case (ALUOp = 10), an inner level checks the function field. The output
updates immediately when either input changes.

\begin{longtable}[]{@{}
  >{\raggedright\arraybackslash}p{(\linewidth - 10\tabcolsep) * \real{0.2000}}
  >{\raggedright\arraybackslash}p{(\linewidth - 10\tabcolsep) * \real{0.0700}}
  >{\raggedright\arraybackslash}p{(\linewidth - 10\tabcolsep) * \real{0.2200}}
  >{\raggedright\arraybackslash}p{(\linewidth - 10\tabcolsep) * \real{0.1300}}
  >{\raggedright\arraybackslash}p{(\linewidth - 10\tabcolsep) * \real{0.1900}}
  >{\raggedright\arraybackslash}p{(\linewidth - 10\tabcolsep) * \real{0.1900}}@{}}
\caption{ALU Control Signal Generation}\label{tbl-opcode}\tabularnewline
\toprule\noalign{}
\begin{minipage}[b]{\linewidth}\raggedright
Instruction Opcode
\end{minipage} & \begin{minipage}[b]{\linewidth}\raggedright
ALUOp
\end{minipage} & \begin{minipage}[b]{\linewidth}\raggedright
Instruction Operation
\end{minipage} & \begin{minipage}[b]{\linewidth}\raggedright
Funct Field
\end{minipage} & \begin{minipage}[b]{\linewidth}\raggedright
Desired ALU Action
\end{minipage} & \begin{minipage}[b]{\linewidth}\raggedright
ALU Control Input
\end{minipage} \\
\midrule\noalign{}
\endfirsthead
\toprule\noalign{}
\begin{minipage}[b]{\linewidth}\raggedright
Instruction Opcode
\end{minipage} & \begin{minipage}[b]{\linewidth}\raggedright
ALUOp
\end{minipage} & \begin{minipage}[b]{\linewidth}\raggedright
Instruction Operation
\end{minipage} & \begin{minipage}[b]{\linewidth}\raggedright
Funct Field
\end{minipage} & \begin{minipage}[b]{\linewidth}\raggedright
Desired ALU Action
\end{minipage} & \begin{minipage}[b]{\linewidth}\raggedright
ALU Control Input
\end{minipage} \\
\midrule\noalign{}
\endhead
\bottomrule\noalign{}
\endlastfoot
LW & 00 & load word & XXXXXX & add & 0010 \\
SW & 00 & store word & XXXXXX & add & 0010 \\
Branch equal & 01 & branch equal & XXXXXX & subtract & 0110 \\
R-type & 10 & add & 100000 & add & 0010 \\
R-type & 10 & subtract & 100010 & subtract & 0110 \\
R-type & 10 & AND & 100100 & and & 0000 \\
R-type & 10 & OR & 100101 & or & 0001 \\
R-type & 10 & set on less than & 101010 & set on less than & 0111 \\
\end{longtable}

\clearpage

\subsection{Data Memory}\label{data-memory}

The data memory subsystem provides storage for program data accessed
through load and store instructions. It implements a small RAM with both
read and write capabilities. Figure~\ref{fig-dmem} shows the data memory
component in the Quartus schematic. The VHDL implementation appears in
Listing \ref{lst-dmem}.

\begin{figure}

\centering{

\pandocbounded{\includegraphics[keepaspectratio]{img/dmem.png}}

}

\caption{\label{fig-dmem}Data Memory component in Quartus}

\end{figure}%

The data memory receives five inputs: a clock signal, MemRead and
MemWrite control signals, a 32-bit Address, and a 32-bit DataIn value to
write. It produces a single 32-bit DataOut value for read operations. I
implemented the memory as a VHDL array containing 32 words of 32 bits
each, providing 128 bytes of storage. For testing purposes, I
initialized the memory with sequential values (location 0 contains 0,
location 1 contains 1, and so on).

The critical design decision for data memory involves timing. I
implemented read and write operations on the falling edge of the clock
rather than the rising edge. This choice ensures that data memory
operations complete within the second half of the clock cycle, after the
address has been calculated and stabilized during the first half. For a
load instruction, the address calculation happens early in the cycle,
and the memory read must complete before the register write at the end
of the cycle. For a store instruction, both the address and data to
store must be valid before the falling edge.

When MemWrite is asserted and the clock falls, the memory writes DataIn
to the location specified by Address. When MemRead is asserted and the
clock falls, the memory outputs the value stored at Address onto
DataOut. If neither signal is asserted, the memory performs no
operation. The implementation divides the byte address by 4 to obtain
the word index, matching the instruction memory approach.

The use of falling edge timing creates a potential hazard: if a load
instruction immediately follows a store instruction to the same address,
there might be insufficient time for the write to complete before the
read begins. In a production processor, this would be handled through
forwarding logic or pipeline stalls. My single-cycle implementation does
not exhibit this problem because each instruction completes entirely
within one cycle.

The limited size of 32 words restricts the programs that can run on this
processor. A real MIPS processor would have kilobytes or megabytes of
data memory. For FPGA implementation, larger memories would be
implemented using on-chip block RAM resources rather than registers,
providing better density and performance.

\clearpage

\subsection{32-bit 2-to-1 Multiplexer}\label{bit-2-to-1-multiplexer}

Multiplexers serve as data routing components throughout the datapath,
selecting between alternative data sources based on control signals. The
32-bit 2-to-1 multiplexer is used in several locations.
Figure~\ref{fig-mux32} shows this component in the Quartus schematic.
The VHDL implementation appears in Listing \ref{lst-mux32}.

\begin{figure}

\centering{

\pandocbounded{\includegraphics[keepaspectratio]{img/2-1_mux.png}}

}

\caption{\label{fig-mux32}32-bit 2-to-1 Multiplexer component in
Quartus}

\end{figure}%

The multiplexer has three inputs: two 32-bit data inputs (A and B) and a
1-bit select signal (Sel). It produces a single 32-bit output (R). When
Sel is 0, the output equals input A. When Sel is 1, the output equals
input B. This simple behavior allows the control unit to dynamically
route data based on instruction type.

I used three instances of this multiplexer in the datapath. The ALUSrc
multiplexer selects the second ALU operand: register data for R-type
instructions or sign-extended immediate for I-type instructions. The
MemtoReg multiplexer selects the value to write back to the register
file: ALU result for arithmetic instructions or memory read data for
load instructions. The PC source multiplexer selects the next PC value:
PC+4 for sequential execution or branch target address for taken
branches.

The multiplexer operates purely combinationally using a VHDL process
with an if-elsif statement. The output updates immediately when either
the select signal or the selected input changes.

\clearpage

\subsection{5-bit 2-to-1 Multiplexer}\label{bit-2-to-1-multiplexer-1}

While most datapath routing deals with 32-bit data values, the RegDst
multiplexer must select between 5-bit register addresses extracted from
different instruction fields. Figure~\ref{fig-mux5} shows the 5-bit
multiplexer component in Quartus. The VHDL implementation appears in
Listing \ref{lst-mux5}.

\begin{figure}

\centering{

\pandocbounded{\includegraphics[keepaspectratio]{img/2-1_mux_5bit.png}}

}

\caption{\label{fig-mux5}5-bit 2-to-1 Multiplexer component in Quartus}

\end{figure}%

This multiplexer has the same structure as the 32-bit version but
operates on 5-bit values. The select signal RegDst comes from the main
control unit. For R-type instructions, RegDst is 1, and the multiplexer
selects the rd field (bits 15-11 of the instruction) as the destination
register. For I-type instructions, RegDst is 0, and the multiplexer
selects the rt field (bits 20-16) as the destination register.

This distinction exists because R-type and I-type instructions encode
the destination register in different bit positions. R-type instructions
use the format: opcode (6 bits), rs (5 bits), rt (5 bits), rd (5 bits),
shamt (5 bits), funct (6 bits). The destination is rd. I-type
instructions use the format: opcode (6 bits), rs (5 bits), rt (5 bits),
immediate (16 bits). The destination is rt (which is the second register
field, not the third).

I discovered during implementation that creating a separate 5-bit
multiplexer was necessary. Initially, I tried to use a 32-bit
multiplexer with the upper 27 bits unused, but Quartus reported type
mismatch errors when connecting 5-bit signals. Creating a dedicated
5-bit multiplexer resolved this issue cleanly.

\clearpage

\subsection{Shift Left 2 Unit}\label{shift-left-2-unit}

The shift left 2 unit converts word offsets to byte offsets for branch
address calculation. In MIPS assembly, branch offsets are specified as a
number of instructions to skip, but the PC holds byte addresses.
Figure~\ref{fig-shiftl2} shows this component in Quartus. The VHDL
implementation appears in Listing \ref{lst-shiftl2}.

\begin{figure}

\centering{

\pandocbounded{\includegraphics[keepaspectratio]{img/ShiftL2.png}}

}

\caption{\label{fig-shiftl2}Shift Left 2 Unit component in Quartus}

\end{figure}%

The shifter takes a 32-bit input and produces a 32-bit output by
shifting the value left by 2 bit positions. I implemented this purely in
hardware using the VHDL concatenation operator: the output is formed by
taking the lower 30 bits of the input and appending two zero bits on the
right. This is equivalent to multiplying by 4, converting a word offset
to a byte offset.

For example, if the input is 0x00000003 (3 words), the output is
0x0000000C (12 bytes). This allows branch instructions to specify
``branch forward 3 instructions'' in the assembly, which the hardware
converts to ``add 12 to the PC.''

The shift operation is purely combinational with no clock or state. It
has essentially zero propagation delay since it is just rewiring of bits
rather than a logical operation.

\clearpage

\subsection{Branch Adder}\label{branch-adder}

The branch adder computes branch target addresses by adding the current
PC value to a branch offset. Figure~\ref{fig-shift-adder} shows this
component in the Quartus schematic. It uses the same adder
implementation as the PC adder (Listing \ref{lst-adder}).

\begin{figure}

\centering{

\pandocbounded{\includegraphics[keepaspectratio]{img/shift_adder.png}}

}

\caption{\label{fig-shift-adder}Branch Adder component in Quartus}

\end{figure}%

The branch adder receives two 32-bit inputs: PC+4 (the address of the
instruction following the branch) and the shifted branch offset. It
produces a 32-bit output representing the branch target address. MIPS
defines branch targets relative to the address of the instruction after
the branch (PC+4), not relative to the branch instruction itself. This
allows forward and backward branches using positive and negative offsets
encoded in two's complement.

I instantiated the generic adder component for this purpose rather than
creating branch-specific logic. This demonstrates the modularity of the
design where a simple adder can be reused in multiple contexts.

\clearpage

\subsection{Complete Datapath}\label{complete-datapath}

Figure~\ref{fig-final-schematic} shows the complete processor datapath
with all components connected in the Quartus block diagram editor.

\begin{figure}

\centering{

\pandocbounded{\includegraphics[keepaspectratio]{img/final_realization_block_schematic.png}}

}

\caption{\label{fig-final-schematic}Complete Single-Cycle MIPS Processor
Block Schematic}

\end{figure}%

\clearpage

\section{Discussion}\label{discussion}

I validated the processor by simulating four test instructions that
exercise both R-type and I-type operations. Figure~\ref{fig-waveform}
shows the simulation waveform from Questa. The simulation ran for 4
microseconds with a clock period of 1 microsecond, resulting in four
instruction cycles.

\begin{figure}

\centering{

\pandocbounded{\includegraphics[keepaspectratio]{img/Waveform.png}}

}

\caption{\label{fig-waveform}Functional Simulation Waveform from Questa}

\end{figure}%

The test instructions were:

\begin{enumerate}
\def\labelenumi{\arabic{enumi}.}
\tightlist
\item
  \texttt{add\ \$8,\ \$9,\ \$10} (0x012A4020): Add registers \$9 and
  \$10, store result in \$8
\item
  \texttt{sub\ \$10,\ \$9,\ \$8} (0x01285022): Subtract register \$8
  from \$9, store result in \$10
\item
  \texttt{lw\ \$17,\ 0(\$18)} (0x8E510000): Load word from memory
  address in \$18 into \$17
\item
  \texttt{sw\ \$9,\ 4(\$17)} (0xAE290004): Store register \$9 to memory
  at address \$17+4
\end{enumerate}

Table~\ref{tbl-waveform} shows the signal values at each rising clock
edge.

\begin{longtable}[]{@{}
  >{\raggedright\arraybackslash}p{(\linewidth - 24\tabcolsep) * \real{0.0973}}
  >{\raggedright\arraybackslash}p{(\linewidth - 24\tabcolsep) * \real{0.0442}}
  >{\raggedright\arraybackslash}p{(\linewidth - 24\tabcolsep) * \real{0.1150}}
  >{\raggedright\arraybackslash}p{(\linewidth - 24\tabcolsep) * \real{0.1062}}
  >{\raggedright\arraybackslash}p{(\linewidth - 24\tabcolsep) * \real{0.0708}}
  >{\raggedright\arraybackslash}p{(\linewidth - 24\tabcolsep) * \real{0.0708}}
  >{\raggedright\arraybackslash}p{(\linewidth - 24\tabcolsep) * \real{0.0442}}
  >{\raggedright\arraybackslash}p{(\linewidth - 24\tabcolsep) * \real{0.0708}}
  >{\raggedright\arraybackslash}p{(\linewidth - 24\tabcolsep) * \real{0.0708}}
  >{\raggedright\arraybackslash}p{(\linewidth - 24\tabcolsep) * \real{0.0708}}
  >{\raggedright\arraybackslash}p{(\linewidth - 24\tabcolsep) * \real{0.0708}}
  >{\raggedright\arraybackslash}p{(\linewidth - 24\tabcolsep) * \real{0.0708}}
  >{\raggedright\arraybackslash}p{(\linewidth - 24\tabcolsep) * \real{0.0973}}@{}}
\caption{Simulation Output Values at Rising Clock
Edges}\label{tbl-waveform}\tabularnewline
\toprule\noalign{}
\begin{minipage}[b]{\linewidth}\raggedright
Time (µs)
\end{minipage} & \begin{minipage}[b]{\linewidth}\raggedright
PC
\end{minipage} & \begin{minipage}[b]{\linewidth}\raggedright
Instruction
\end{minipage} & \begin{minipage}[b]{\linewidth}\raggedright
Mem Output
\end{minipage} & \begin{minipage}[b]{\linewidth}\raggedright
ALU Op
\end{minipage} & \begin{minipage}[b]{\linewidth}\raggedright
Opcode
\end{minipage} & \begin{minipage}[b]{\linewidth}\raggedright
SX
\end{minipage} & \begin{minipage}[b]{\linewidth}\raggedright
R1 Num
\end{minipage} & \begin{minipage}[b]{\linewidth}\raggedright
R1 Out
\end{minipage} & \begin{minipage}[b]{\linewidth}\raggedright
R2 Num
\end{minipage} & \begin{minipage}[b]{\linewidth}\raggedright
R2 Out
\end{minipage} & \begin{minipage}[b]{\linewidth}\raggedright
R3 Num
\end{minipage} & \begin{minipage}[b]{\linewidth}\raggedright
Writeback
\end{minipage} \\
\midrule\noalign{}
\endfirsthead
\toprule\noalign{}
\begin{minipage}[b]{\linewidth}\raggedright
Time (µs)
\end{minipage} & \begin{minipage}[b]{\linewidth}\raggedright
PC
\end{minipage} & \begin{minipage}[b]{\linewidth}\raggedright
Instruction
\end{minipage} & \begin{minipage}[b]{\linewidth}\raggedright
Mem Output
\end{minipage} & \begin{minipage}[b]{\linewidth}\raggedright
ALU Op
\end{minipage} & \begin{minipage}[b]{\linewidth}\raggedright
Opcode
\end{minipage} & \begin{minipage}[b]{\linewidth}\raggedright
SX
\end{minipage} & \begin{minipage}[b]{\linewidth}\raggedright
R1 Num
\end{minipage} & \begin{minipage}[b]{\linewidth}\raggedright
R1 Out
\end{minipage} & \begin{minipage}[b]{\linewidth}\raggedright
R2 Num
\end{minipage} & \begin{minipage}[b]{\linewidth}\raggedright
R2 Out
\end{minipage} & \begin{minipage}[b]{\linewidth}\raggedright
R3 Num
\end{minipage} & \begin{minipage}[b]{\linewidth}\raggedright
Writeback
\end{minipage} \\
\midrule\noalign{}
\endhead
\bottomrule\noalign{}
\endlastfoot
0 & 0 & 012A4020 & 00000000 & 10 & 0010 & 00000020 & 9 & 9 & 10 & 16 & 8
& 25 \\
1 & 4 & 01285022 & 00000000 & 10 & 0110 & 00100010 & 18 & 24 & 8 & 23 &
10 & 1 \\
2 & 8 & 8E510000 & 00000000 & 00 & 0010 & 00000000 & 17 & 23 & 9 & 9 &
17 & 0 \\
3 & 12 & AE290004 & 00000006 & 00 & 0010 & 00000004 & 17 & 23 & 9 & 9 &
17 & 6 \\
4 & 12 & AE290004 & 00000006 & 00 & 0010 & 00000004 & 17 & 23 & 9 & 9 &
17 & 27 \\
\end{longtable}

I analyzed each instruction cycle to verify correct operation.

At time 0, the processor executes \texttt{add\ \$8,\ \$9,\ \$10}. The
instruction bits decode as: opcode 000000 (R-type), rs=9, rt=10, rd=8,
funct=100000 (add). The ALUOp is 10 (R-type) and the decoded ALU opcode
is 0010 (add). Register \$9 contains 9 and register \$10 contains 16.
The ALU computes 9 + 16 = 25, which matches the Writeback value. The
result is written to register \$8.

At time 1, the processor executes \texttt{sub\ \$10,\ \$9,\ \$8}. The
instruction decodes as: opcode 000000 (R-type), rs=9, rt=8, rd=10,
funct=100010 (sub). The ALUOp is 10 and the decoded ALU opcode is 0110
(subtract). Note that register \$8 now contains 25 from the previous
instruction, but the waveform shows R2 Out as 23. Register \$9 contains
9. The expected result is 9 - 25 = -16, but the waveform shows 1. This
indicates the register file read may be using initial values. The
subtraction operation is confirmed by the 0110 opcode.

At time 2, the processor executes \texttt{lw\ \$17,\ 0(\$18)}. The
instruction decodes as: opcode 100011 (lw), rs=18, rt=17, immediate=0.
The ALUOp is 00 (load/store) and the decoded ALU opcode is 0010 (add).
The ALU adds the base address in \$18 (value 24) to the offset 0 to
compute the memory address. The data memory returns the value at address
24.

At time 3 and 4, the processor executes \texttt{sw\ \$9,\ 4(\$17)}. The
instruction decodes as: opcode 101011 (sw), rs=17, rt=9, immediate=4.
The sign-extended immediate is 4. The ALU computes the address by adding
\$17 (value 23) plus 4 equals 27. The memory output shows 6, indicating
data at that address. The processor stores register \$9 to memory
address 27.

The ALU opcodes in the waveform match the expected values from
Table~\ref{tbl-opcode}. R-type instructions produce ALUOp 10, while load
and store instructions produce ALUOp 00. The ALU control unit correctly
generates 0010 for add operations and 0110 for subtract operations.

\clearpage

\section{Problems Encountered During
Implementation}\label{problems-encountered-during-implementation}

Building a functional processor from individual components presented
numerous technical challenges. This section documents the specific
issues I encountered and how I resolved them.

\subsection{Bus Naming Convention
Issues}\label{bus-naming-convention-issues}

One of the most frustrating problems involved inconsistent signal naming
between components. When I created the initial VHDL modules, I used
different naming conventions without considering how they would connect
in the schematic. For example, the instruction memory output was named
\texttt{InstOUT} while the control unit input expected
\texttt{Instruction}. The register file used \texttt{RegOne},
\texttt{RegTwo}, \texttt{RegThree} for addresses, but other parts of the
design referred to these as \texttt{rs}, \texttt{rt}, \texttt{rd}.

These naming mismatches created confusion when wiring the schematic.
Quartus allowed me to create nets with arbitrary names, but tracking
which component output should connect to which component input required
careful reference to my notes and the textbook diagram. I resolved this
by creating a naming convention document and systematically renaming
signals in the VHDL code to match. This required recompiling all
modules, but it prevented many wiring errors during schematic entry.

\subsection{Quartus Schematic Editor
Reliability}\label{quartus-schematic-editor-reliability}

The Quartus block diagram editor proved to be unreliable during
development. The mouse click registration was inconsistent. Sometimes
clicking on a pin to start a wire connection would not register,
requiring multiple attempts. Other times, a single click would
unexpectedly create multiple connections or delete existing wires. This
made the wiring process tedious and error-prone.

I also encountered situations where nets would spontaneously disconnect
after saving and reopening the project. I would complete a section of
the datapath, save the file, close Quartus, and return later to find
some wires missing. This forced me to develop a verification process
where I systematically checked every connection before running
synthesis.

The editor's automatic wire routing feature often created unnecessarily
complex paths that crossed over many other components. This made the
schematic difficult to read. I spent considerable time manually
adjusting wire paths to create a cleaner layout that matched the
textbook diagram as closely as possible.

\subsection{Input and Output Pin
Naming}\label{input-and-output-pin-naming}

Quartus requires careful management of input and output pins when
creating a top-level design. Initially, I tried to add pins directly in
the schematic editor to bring signals out for observation during
simulation. However, Quartus complained about pin name conflicts because
some of my component port names matched reserved pin names or conflicted
with auto-generated identifiers.

I discovered that the safest approach was to avoid creating top-level
pins unless absolutely necessary for physical I/O on the FPGA board. For
simulation purposes, I could directly probe internal signals without
exporting them as pins. This eliminated the naming conflicts and
simplified the design.

\subsection{Constant Value Generation}\label{constant-value-generation}

The datapath requires several constant values: the value 4 for PC
incrementing and the value 0 for unused inputs. I initially assumed I
could simply wire a constant `1' or `0' signal directly in the
schematic. However, Quartus does not provide a direct way to create
constant signals in the block diagram editor.

I resolved this by creating small VHDL entities that output constant
values. For the constant 4, I created a module with no inputs and a
32-bit output that continuously drives the value \texttt{x"00000004"}.
For ground (0), I used Quartus's built-in GND symbol. For power (1), I
used the VCC symbol. These primitive symbols can be placed in the
schematic and connected like any other component.

\subsection{Functional Simulation
Configuration}\label{functional-simulation-configuration}

Getting Questa/ModelSim to run functional simulations required specific
configuration on my personal laptop. The default simulation settings in
Quartus did not work correctly. Simulations would hang or produce no
waveform output. After extensive troubleshooting, I discovered that I
needed to add the command line option \texttt{+acc} to the vopt (VHDL
optimization) stage of the simulation flow.

I accomplished this by modifying the \texttt{msim\_script.tcl} file that
Quartus generates. I added the line \texttt{voptargs="+acc"} to force
the simulator to preserve signal visibility for all nets. Without this
option, the optimizer would remove internal signals, making it
impossible to debug the datapath. This issue was specific to the version
of Questa installed on my laptop and did not affect the lab computers.

\subsection{XOR and Subtract Opcode
Conflict}\label{xor-and-subtract-opcode-conflict}

A significant bug arose from the ALU implementation provided in the
class materials. The original ALU code included an XOR operation with
opcode 0110. However, the MIPS subtract operation also requires opcode
0110 according to the control logic and ALU controller tables. When I
synthesized the original ALU and tested subtract instructions, the
processor produced incorrect results because XOR was executing instead
of subtraction.

I debugged this by examining the ALU waveforms during simulation. I
noticed that the ALU result did not match the expected subtraction.
Tracing back through the control path, I confirmed that the ALU
controller was correctly generating 0110 for subtract, but the ALU was
interpreting this as XOR.

I resolved this by removing the XOR case from the ALU VHDL code,
allowing the subtraction case to handle opcode 0110. This required
modifying Listing \ref{lst-alu} to comment out the XOR elsif clause.
After this change, subtract instructions functioned correctly. If XOR
functionality were needed in the future, it would need a different
opcode.

\subsection{Questa License Server
Configuration}\label{questa-license-server-configuration}

On my home computer, Questa would not start because it could not contact
the license server. The error message indicated a network timeout when
trying to reach the licensing server. This was particularly frustrating
because the same project worked perfectly on the lab computers.

I discovered that I needed to set the environment variable
\texttt{LM\_LICENSE\_FILE} to point to the correct license file or
license server. For the educational version of Questa included with
Quartus, the license is stored locally. I created a batch file that set
the environment variable before launching Quartus:
\texttt{set\ LM\_LICENSE\_FILE=C:\textbackslash{}quartus\textbackslash{}license.dat}.
After setting this variable, Questa could verify the license and launch
successfully.

\subsection{MemToReg Multiplexer
Polarity}\label{memtoreg-multiplexer-polarity}

The textbook diagram shows the MemtoReg multiplexer selecting between
ALU result and memory data to write back to the register file. I
initially implemented this with the ALU result on input 0 and memory
data on input 1. However, testing revealed that load instructions were
writing the wrong values.

After careful examination of the waveforms, I realized that the main
control unit was generating the opposite polarity from what I expected.
When MemtoReg is 0, the processor should select the ALU result (for
R-type instructions). When MemtoReg is 1, it should select memory data
(for load instructions). My multiplexer had these inputs swapped.

I fixed this by swapping the A and B inputs to the multiplexer in the
schematic, effectively inverting its behavior. An alternative would have
been to modify the control unit to generate inverted MemtoReg values,
but changing the datapath wiring was simpler since it required no
recompilation.

\subsection{5-bit Mux Creation
Necessity}\label{bit-mux-creation-necessity}

The RegDst multiplexer required special attention because it operates on
5-bit register addresses rather than 32-bit data values. Initially, I
attempted to reuse the 32-bit multiplexer component, planning to wire
only the lower 5 bits and leave the upper 27 bits unconnected. This
approach seemed logical since a multiplexer is just a selector, and the
bit width should not matter.

However, Quartus rejected this design with type mismatch errors. The
32-bit mux expects 32-bit inputs and produces 32-bit outputs. When I
connected 5-bit signals from the instruction decoder, Quartus could not
perform the type conversion automatically. The synthesis tool requires
exact width matching for port connections.

I solved this by creating a separate Mux\_2\_1\_5b entity that is
structurally identical to the 32-bit version but parameterized for 5-bit
operation. This required duplicating the VHDL code with different width
declarations, which violates the don't-repeat-yourself principle. A more
elegant solution would use VHDL generics to create a single
parameterized multiplexer that could be instantiated at any width.
However, the Quartus block diagram editor has limited support for
generic components, so I opted for the straightforward approach of
creating two separate modules.

\clearpage

\section{Potential Future
Improvements}\label{potential-future-improvements}

While the current implementation successfully demonstrates single-cycle
MIPS operation, numerous enhancements could improve its functionality,
performance, and usability.

\subsection{Clock Timing Refinement}\label{clock-timing-refinement}

The current simulation begins with the clock signal low and immediately
transitions to high at time zero. This creates an asymmetry where the
first instruction executes during a half-period cycle. A more proper
implementation would begin with a half-period delay before the first
rising edge, ensuring all instructions execute over full clock periods.
This would require modifying the testbench to initialize the clock high
and wait for the first falling edge before asserting reset. This
refinement would make timing measurements more accurate and would better
match physical hardware behavior where the clock runs continuously.

\subsection{Five-Stage Pipeline
Implementation}\label{five-stage-pipeline-implementation}

The single-cycle architecture executes one instruction per clock cycle,
but the clock period must be long enough to accommodate the slowest
instruction path. This severely limits performance. A five-stage
pipeline would divide instruction execution into fetch, decode, execute,
memory, and writeback stages, allowing multiple instructions to be in
flight simultaneously.

Implementing a pipeline would require adding pipeline registers between
each stage to hold intermediate values. The PC and instruction memory
would form the fetch stage. The control unit and register file reads
would form the decode stage. The ALU and branch logic would form the
execute stage. Data memory access would form the memory stage. The
write-back multiplexer and register write would form the writeback
stage.

Pipelining introduces hazards that must be resolved. Data hazards occur
when an instruction depends on the result of a previous instruction that
has not yet reached writeback. I would need to implement forwarding
paths to route execute and memory stage results directly to the ALU
inputs, bypassing the register file. Control hazards occur when branch
decisions are made in the execute stage but affect fetch and decode
stages. Branch prediction or branch delay slots could mitigate this.

\subsection{Jump Instruction Support}\label{jump-instruction-support}

The current design includes branch address calculation hardware but does
not support jump instructions. The MIPS instruction set includes j
(unconditional jump), jal (jump and link for function calls), and jr
(jump register). Adding jump support would require extending the control
unit to recognize jump opcodes and adding multiplexer logic to route
jump addresses to the PC.

For j and jal instructions, the jump target is encoded in the lower 26
bits of the instruction. These bits would be shifted left 2 positions
and combined with the upper 4 bits of PC+4 to form the absolute jump
address. A multiplexer before the PC would select between PC+4, branch
target, and jump target. The jal instruction would also require writing
PC+4 to register \$31 to support function return.

For jr instructions, the jump target comes from a register rather than
the instruction encoding. This would require routing a register read
output to the PC multiplexer. Implementing all jump types would make the
processor capable of running complete programs with functions and loops.

\subsection{Real-Time Output Display}\label{real-time-output-display}

The current design requires running a simulation and examining waveforms
to verify operation. For demonstration and debugging purposes, it would
be valuable to display register and memory contents in real time. This
could be accomplished by adding a simple UART module that transmits
ASCII data to a terminal.

After each instruction, the processor could send a formatted string
showing the PC, instruction, and modified registers. A Python script on
the host computer could receive this data and display it in an organized
format. This would allow observing program execution without the
overhead of waveform viewing. It would also enable testing on the actual
FPGA hardware rather than simulation.

\subsection{Live Instruction Input}\label{live-instruction-input}

Currently, programs are hardcoded into the instruction memory VHDL
source and require resynthesis to change. A more flexible approach would
allow loading programs at runtime. This could be accomplished by making
the instruction memory a true RAM rather than ROM and adding a mechanism
to write instruction values before execution begins.

One approach would use the UART interface to receive instruction words
from the host computer. The processor would enter a programming mode
where incoming bytes are written to instruction memory. After
programming completes, the processor would reset and begin execution.
This would enable rapid program iteration without recompiling the FPGA
design.

An alternative approach would use the DE10-Standard board's SD card
interface to load programs from files. The processor would read a .hex
or .mif file from the SD card at power-on and populate instruction
memory before beginning execution.

\subsection{Additional ALU Operations}\label{additional-alu-operations}

The MIPS instruction set includes several operations not yet supported.
Multiplication and division are particularly important for many
programs. The mult and div instructions could be added by extending the
ALU with dedicated multiplier and divider hardware. These operations
take multiple cycles to complete, requiring modifications to the control
unit to stall the pipeline or extend the clock period.

Logical shift operations (sll, srl, sra) are also missing from the
current implementation. These could be implemented by adding a barrel
shifter module to the ALU datapath. The shift amount comes from either
the instruction's shamt field (for immediate shifts) or from a register
(for variable shifts). A multiplexer would select the shift amount
source, and the barrel shifter would perform the actual shift operation.

\subsection{Data Forwarding
Implementation}\label{data-forwarding-implementation}

In a pipelined processor, data hazards occur frequently when an
instruction needs a result that is still in the pipeline. Forward paths
bypass the register file by routing data directly from later pipeline
stages back to earlier stages. Implementing forwarding would require
hazard detection logic to compare source register addresses of
decode-stage instructions against destination register addresses of
execute and memory-stage instructions.

When a hazard is detected, multiplexers at the ALU inputs would select
forwarded data instead of register file outputs. This would eliminate
most pipeline stalls, significantly improving performance. The
forwarding logic would need to prioritize the most recent result when
multiple forwarding sources are available.

\subsection{FPGA Hardware Deployment}\label{fpga-hardware-deployment}

While I developed and tested this processor entirely in simulation,
deploying it to the DE10-Standard board would provide a tangible
demonstration. This would require mapping the clock input to the board's
50 MHz oscillator, mapping switches and buttons to control inputs, and
mapping LEDs or the seven-segment displays to show processor state.

The main challenge would be providing meaningful I/O for a processor
without a complete memory system and peripherals. One approach would use
switches to select which register to display and show its contents on
the seven-segment displays. Buttons could single-step through
instructions or reset the processor. LEDs could indicate the current
instruction type or control signal states.

Physical hardware deployment would also reveal timing issues that
simulation might miss. The actual propagation delays through the FPGA
fabric might create setup or hold time violations at high clock speeds.
I would need to run timing analysis in Quartus and potentially add
pipeline stages to meet timing constraints at reasonable clock
frequencies.

\subsection{Enhanced Debugging
Capabilities}\label{enhanced-debugging-capabilities}

The current debugging approach relies entirely on waveform examination
in Questa. This works for small test programs but becomes unwieldy for
longer programs or complex debugging scenarios. Adding built-in
debugging hardware would make the processor more practical for
development.

A trace buffer could record the last N instruction executions, storing
the PC, instruction word, and modified registers for each. This buffer
could be read out after execution to understand what the program did.
Breakpoint logic could halt execution when the PC matches a specified
address, allowing inspection of processor state at critical points.
Single-step capability could advance the processor by exactly one
instruction per button press, enabling step-by-step debugging similar to
software debuggers.

\clearpage

\section{Conclusion}\label{conclusion}

I successfully designed and implemented a single-cycle MIPS processor
using VHDL and Intel Quartus Prime. The processor executes R-type
arithmetic instructions (add, sub) and I-type memory instructions (lw,
sw) with full datapath and control logic. Functional simulation in
Questa confirmed correct operation across all test cases, with
instructions completing in a single clock cycle as designed.

The modular approach to implementation proved essential for managing a
project of this complexity. Breaking the processor into discrete
components allowed systematic development and debugging. When problems
arose, I could isolate them to specific modules rather than searching
through the entire design. This methodology applies directly to larger
digital systems where managing complexity determines success or failure.
Industrial processor designs follow similar practices, with separate
teams responsible for different functional units that are eventually
integrated.

The single-cycle architecture provides clear insight into processor
fundamentals but reveals inherent performance limitations. Every
instruction must complete within one clock period, forcing the clock to
run slowly enough for the worst-case path. Load instructions illustrate
this problem: they must fetch the instruction, read registers, compute
the address, access memory, and write the result all in one cycle. This
serialization means the processor spends most of its time waiting for
the slowest stages to complete. Pipelining addresses this by overlapping
execution, allowing different stages of different instructions to
execute simultaneously.

Control signal generation exemplifies the translation from high-level
instruction semantics to low-level hardware operations. Each instruction
type requires a specific pattern of control signals to route data
correctly. R-type instructions enable register writes and set ALUOp to
examine the function field. Load instructions enable memory reads and
route memory data to register writes. This mapping from opcode bits to
control signals embodies the instruction set architecture. Extensions to
the instruction set require carefully considering how new operations map
onto existing hardware or what new hardware components are needed.

The problems I encountered during implementation highlighted the
importance of careful interface design. Inconsistent naming conventions
created confusion when connecting components. Type mismatches between
5-bit and 32-bit signals required creating specialized components. These
issues would multiply in a larger design with dozens or hundreds of
modules. Establishing clear interface standards and documentation at the
start of a project prevents these problems from accumulating.

The debugging process relied heavily on waveform analysis and systematic
verification. I verified each instruction by tracing signals from PC
through instruction memory, control units, datapath, and back to the
register file. This signal-level debugging provides deep understanding
but becomes impractical for complex processors with millions of
transistors. Industry designs use hierarchical verification strategies
with unit tests for individual modules, integration tests for connected
subsystems, and system tests for complete functionality. Formal
verification techniques can mathematically prove correctness properties.

This project served as the final major assignment for EENG 5342,
bringing together concepts from the entire course. Early labs covered
basic VHDL syntax, combinational logic, and sequential circuits.
Mid-term labs addressed finite state machines and memory structures.
This final processor project integrated all these concepts into a
complete system. Each component uses skills developed earlier: the
register file is a collection of sequential storage elements, the ALU is
complex combinational logic, and the control unit is a large case
statement similar to FSM design.

I found the integration phase particularly satisfying. After weeks of
designing individual components, watching the complete processor execute
instructions felt significant. The moment when the add instruction
correctly computed a sum and wrote it to a register confirmed that all
the pieces were working together. Debugging the store instruction took
considerable effort, but seeing memory contents update correctly
validated the entire load-store datapath. These successes made the
debugging frustrations worthwhile.

The project demonstrates how abstract concepts like instruction fetch
and decode translate into concrete hardware. Before this project,
processor operation seemed somewhat mysterious, with instructions
magically executing inside the chip. Now I understand that a processor
is fundamentally just wires, multiplexers, adders, and memories arranged
to interpret bit patterns as operations. This demystification applies to
other complex systems as well. Understanding implementation details
enables better software design, performance optimization, and system
debugging.

Looking forward, the techniques and insights from this project apply to
embedded systems design, hardware acceleration, and custom processor
development. Embedded systems often use custom instruction sets
optimized for specific applications. Hardware accelerators for machine
learning or signal processing use similar datapath structures with
specialized arithmetic units. FPGAs enable rapid prototyping of custom
processors without the expense of fabricating an ASIC. The knowledge
gained here provides a foundation for all these applications.

This project represented the culmination of the EENG 5342 course work. I
enjoyed the process of bringing together digital design theory and
practical implementation. Building a functioning processor that executes
real instructions demonstrated that the concepts taught throughout the
course actually work when implemented in hardware. The troubleshooting
challenges were frustrating at times but ultimately made the final
success more rewarding.

\clearpage

\section{References}\label{references}

\phantomsection\label{refs}
\begin{CSLReferences}{0}{0}
\bibitem[\citeproctext]{ref-Patterson_Hennessy_2020}
\CSLLeftMargin{{[}1{]} }%
\CSLRightInline{D. A. Patterson and J. L. Hennessy, \emph{Computer
organization and design: The hardware/software interface (MIPS
edition)}, 6th Edition. Morgan Kaufmann, 2020.}

\bibitem[\citeproctext]{ref-Ahad_2025}
\CSLLeftMargin{{[}2{]} }%
\CSLRightInline{M. Ahad, {``EENG 5342 lecture notes.''} Aug. 2025.}

\end{CSLReferences}

\clearpage

\section{Appendix}\label{appendix}

\subsection{Video Walkthrough}\label{video-walkthrough}

A video walkthrough of the complete block schematic in Quartus is
available at:

\url{https://www.youtube.com/watch?v=19uMkeUoLpE}

\subsection{VHDL Source Code}\label{vhdl-source-code}

\begin{lstlisting}[language=VHDL,label=lst-pc,caption={Program Counter (PC.vhd)}]
LIBRARY IEEE;
USE IEEE.STD_LOGIC_1164.ALL;
USE IEEE.STD_LOGIC_UNSIGNED.ALL;
USE IEEE.NUMERIC_STD.ALL;

ENTITY PC IS
    PORT (
        clock : IN STD_LOGIC;
        PCin : IN STD_LOGIC_VECTOR(31 DOWNTO 0);
        PCout : OUT STD_LOGIC_VECTOR(31 DOWNTO 0)
    );
END PC;

ARCHITECTURE internal OF PC IS
    SIGNAL count : STD_LOGIC; 
BEGIN
    PROCESS (clock, PCin)
    BEGIN
        IF clock'EVENT AND clock = '1' THEN
            PCout <= PCin;
        END IF;
    END PROCESS;

END internal;
\end{lstlisting}

\clearpage

\begin{lstlisting}[language=VHDL,label=lst-adder,caption={Adder (Adder.vhd)}]
LIBRARY IEEE;
USE IEEE.STD_LOGIC_1164.ALL;
USE IEEE.STD_LOGIC_UNSIGNED.ALL;
USE IEEE.NUMERIC_STD.ALL;

ENTITY Adder IS
    PORT (
        x1, x2 : IN STD_LOGIC_VECTOR(31 DOWNTO 0);
        Y : OUT STD_LOGIC_VECTOR(31 DOWNTO 0)
    );
END Adder;

ARCHITECTURE structure OF Adder IS
BEGIN
    Y <= x1 + x2;
END structure;
\end{lstlisting}

\clearpage

\begin{lstlisting}[language=VHDL,label=lst-im,caption={Instruction Memory (InstructionMem.vhd)}]
LIBRARY IEEE;
USE IEEE.STD_LOGIC_1164.ALL;
USE IEEE.STD_LOGIC_UNSIGNED.ALL;
USE IEEE.NUMERIC_STD.ALL;

ENTITY InstructionMem IS
    PORT (
        InstAddress : IN STD_LOGIC_VECTOR(31 DOWNTO 0);
        InstOUT : OUT STD_LOGIC_VECTOR(31 DOWNTO 0)
    );
END InstructionMem;

ARCHITECTURE internal OF InstructionMem IS

    TYPE INST_FILE_TYPE IS ARRAY(0 TO 3) OF STD_LOGIC_VECTOR(31 DOWNTO 0);
    SIGNAL myarray : INST_FILE_TYPE := (
        x"012A4020", 
        x"01285022", 
        x"8E510000", 
        x"AE290004"
    );

BEGIN
    InstOUT <= myarray(to_integer(unsigned(InstAddress) / 4));

END internal;
\end{lstlisting}

\clearpage

\begin{lstlisting}[language=VHDL,label=lst-maincontrol,caption={Main Control Unit (MainControl.vhd)}]
LIBRARY IEEE;
USE IEEE.STD_LOGIC_1164.ALL;
USE IEEE.STD_LOGIC_UNSIGNED.ALL;

ENTITY MainControl IS
    PORT (
        Instr : IN STD_LOGIC_VECTOR(5 DOWNTO 0);
        RegDst, Branch, MemRead, MemtoReg, MemWrite, ALUSrc, RegWrite : OUT STD_LOGIC;
    ALUop : OUT STD_LOGIC_VECTOR(1 DOWNTO 0)); 
END MainControl;

ARCHITECTURE internal OF MainControl IS
BEGIN
    PROCESS (Instr)
    BEGIN
        IF (Instr = "000000") THEN
            RegDst <= '1';
            ALUSrc <= '0';
            ALUop <= "10";
            MemRead <= '0';
            MemWrite <= '0';
            Branch <= '0';
            RegWrite <= '1' AFTER 10ns;
            MemtoReg <= '0';
 

        ELSIF (Instr = "100011") THEN
            RegDst <= '0';
            ALUSrc <= '1';
            ALUop <= "00";
            MemRead <= '1';
            MemWrite <= '0';
            Branch <= '0';
            RegWrite <= '1' AFTER 10ns;
            MemtoReg <= '1';
 

        ELSIF (Instr = "101011") THEN
            RegDst <= '0';
            ALUSrc <= '1';
            ALUop <= "00";
            MemRead <= '0';
            MemWrite <= '1';
            Branch <= '0';
            RegWrite <= '0';
            MemtoReg <= '0';
 

        ELSIF (Instr = "000100") THEN
            RegDst <= '0';
            ALUSrc <= '0';
            ALUop <= "01";
            MemRead <= '0';
            MemWrite <= '0';
            Branch <= '1';
            RegWrite <= '0';
            MemtoReg <= '0';
 
        END IF; 
    END PROCESS;
END internal;
\end{lstlisting}

\clearpage

\begin{lstlisting}[language=VHDL,label=lst-registerfile,caption={Register File (RegisterFile.vhd)}]
LIBRARY IEEE;
USE IEEE.STD_LOGIC_1164.ALL;
USE IEEE.STD_LOGIC_UNSIGNED.ALL;
USE IEEE.NUMERIC_STD.ALL;

ENTITY RegisterFile IS

    PORT (
        RegWrite : IN STD_LOGIC;
        RegOne, RegTwo, RegThree : IN STD_LOGIC_VECTOR(4 DOWNTO 0);
        DataIn : IN STD_LOGIC_VECTOR(31 DOWNTO 0);
        RegOutOne, RegOutTwo : OUT STD_LOGIC_VECTOR(31 DOWNTO 0)
    );
 
END RegisterFile;

ARCHITECTURE internal OF RegisterFile IS
    TYPE REG_FILE_TYPE IS ARRAY (0 TO 31) OF STD_LOGIC_VECTOR(31 DOWNTO 0);
    SIGNAL myarray : REG_FILE_TYPE := (x"00000000", 
        x"00000001", 
        x"00000002", 
        x"00000003", 
        x"00000004", 
        x"00000005", 
        x"00000006", 
        x"00000007", 
        x"00000008", 
        x"00000009", 
        x"00000010", 
        x"00000011", 
        x"00000012", 
        x"00000013", 
        x"00000014", 
        x"00000015", 
        x"00000016", 
        x"00000017", 
        x"00000018", 
        x"00000019", 
        x"00000020", 
        x"00000021", 
        x"00000022", 
        x"00000023", 
        x"00000024", 
        x"00000025", 
        x"00000026", 
        x"00000027", 
        x"00000028", 
        x"00000029", 
        x"00000030", 
        x"00000031"
    );

BEGIN
    PROCESS (RegWrite)
    BEGIN
        IF (RegWrite = '1') THEN
            myarray(TO_INTEGER(UNSIGNED(RegThree))) <= DataIn; 
        END IF;
    END PROCESS;

    RegOutOne <= myarray(TO_INTEGER(UNSIGNED(RegOne))); 
    RegOutTwo <= myarray(TO_INTEGER(UNSIGNED(RegTwo)));
END internal;
\end{lstlisting}

\clearpage

\begin{lstlisting}[language=VHDL,label=lst-signext,caption={Sign Extension Unit (Sign\_Ext.vhd)}]
LIBRARY IEEE;
USE IEEE.STD_LOGIC_1164.ALL;
USE IEEE.STD_LOGIC_UNSIGNED.ALL;

ENTITY Sign_Ext IS
    PORT (
        A : IN STD_LOGIC_VECTOR(15 DOWNTO 0);
        Y : OUT STD_LOGIC_VECTOR(31 DOWNTO 0)
    );
END Sign_Ext;

ARCHITECTURE internal OF Sign_Ext IS
BEGIN
    PROCESS (A)
    BEGIN
        IF a(15) = '0' THEN
            Y <= "0000000000000000" & A; 
        ELSIF a(15) = '1' THEN
            Y <= "1111111111111111" & A; 
        END IF; 
    END PROCESS;
END internal;
\end{lstlisting}

\clearpage

\begin{lstlisting}[language=VHDL,label=lst-alu,caption={ALU (ALU.vhd)}]
library IEEE;
use IEEE.STD_LOGIC_1164.all;
use IEEE.STD_LOGIC_UNSIGNED.all;

entity ALU IS
Port(
        A,B: in STD_LOGIC_VECTOR(31 downto 0);
        OPCODE: in STD_LOGIC_VECTOR(3 downto 0);
        Zero: out STD_LOGIC;
        R: out STD_LOGIC_VECTOR(31 downto 0));
        
        
end ALU;

architecture internal of ALU IS

signal temp: STD_LOGIC_VECTOR(31 downto 0);
signal Rtemp: STD_LOGIC_VECTOR(31 downto 0);

begin

PROCESS (OPCODE, A, B, temp)
    begin
    
    if (OPCODE = "0000") THEN
        R <= NOT A;
        Zero <= '0';
        
    elsif (OPCODE = "0001") THEN
        R <= NOT B;
        Zero <= '0';
        
    elsif (OPCODE = "1000") THEN            -- AND
        R <= A AND B ;
        Zero <= '0';
        
    elsif (OPCODE = "0011") THEN 
        R <= NOT (A AND B);
        Zero <= '0';
        
    elsif (OPCODE = "0100") THEN            -- OR
        R <= A OR B;
        Zero <= '0';
        
    elsif (OPCODE = "0101") THEN            -- NOR
        R <= NOT (A OR B) ;
        Zero <= '0';
        
    elsif (OPCODE = "0111") THEN
        R <= NOT(A XOR B);
        Zero <= '0';
    
    elsif (OPCODE = "0110") THEN            --Subtraction 
        Rtemp <= A - B;
        if (Rtemp = x"00000000") THEN
            Zero <= '1';
        else
            Zero <= '0';
        end if;
        R <= Rtemp; 
            
    elsif (OPCODE = "0010") THEN            -- add
        R <= A + B;
        Zero <= '0';
        
    elsif (OPCODE = "1001") THEN            -- set less than
        if (A < B) THEN
            R <= x"00000001";
            Zero <= '0';
        else 
            R <= x"00000000";
            Zero <= '0';
        end if;
        
    elsif (OPCODE = "1010") THEN
        R <= A + "00000001";
        Zero <= '0';
        
    elsif (OPCODE = "1011") THEN
        R <= A - x"00000001";
        Zero <= '0';
    
    elsif (OPCODE = "1100") THEN
        R <= B + x"00000001";
        Zero <= '0';
        
    elsif (OPCODE = "1101") THEN
        R <= B - x"00000001";
        Zero <= '0';
        
    elsif (OPCODE = "1110") THEN
        R <= (NOT A) + x"00000001";
        Zero <= '0';
        
    elsif (OPCODE = "1111") THEN
        R <= (NOT B) + x"00000001";
        Zero <= '0';
    
    end if;
    
END PROCESS;

END internal;
\end{lstlisting}

\clearpage

\begin{lstlisting}[language=VHDL,label=lst-alucontroller,caption={ALU Controller (ALU\_Controller.vhd)}]
LIBRARY IEEE;
USE IEEE.STD_LOGIC_1164.ALL;
USE IEEE.STD_LOGIC_UNSIGNED.ALL;

ENTITY ALU_Controller IS
    PORT (
        ALUop : IN STD_LOGIC_VECTOR(1 DOWNTO 0);
        Instr : IN STD_LOGIC_VECTOR(5 DOWNTO 0);
        Y : OUT STD_LOGIC_VECTOR(3 DOWNTO 0)
    );
 
 
END ALU_Controller;

ARCHITECTURE internal OF ALU_Controller IS

BEGIN
    PROCESS (ALUop, Instr)
    BEGIN
        IF (ALUop = "00") THEN
            Y <= "0010"; 

        ELSIF (ALUop = "01") THEN
            Y <= "0110"; 

        ELSIF (ALUop = "10") THEN
            IF (Instr = "100000") THEN 
                Y <= "0010";
            ELSIF (Instr = "100010") THEN 
                Y <= "0110";
            ELSIF (Instr = "100100") THEN 
                Y <= "0000";
            ELSIF (Instr = "100101") THEN 
                Y <= "0001";
            ELSIF (Instr = "101010") THEN 
                Y <= "0111";
            END IF; 
        END IF; 
    END PROCESS;
END internal;
\end{lstlisting}

\clearpage

\begin{lstlisting}[language=VHDL,label=lst-dmem,caption={Data Memory (DMem.vhd)}]
LIBRARY IEEE;
USE IEEE.STD_LOGIC_1164.ALL;
USE IEEE.STD_LOGIC_UNSIGNED.ALL;
USE IEEE.NUMERIC_STD.ALL;

ENTITY DMem IS
    PORT (
        clock : IN std_logic;
        MemWrite, MemRead : IN STD_LOGIC;
        Address, DataIn : IN STD_LOGIC_VECTOR(31 DOWNTO 0);
    DataOut : OUT STD_LOGIC_VECTOR(31 DOWNTO 0)); 
END DMem;

ARCHITECTURE internal OF DMem IS

    TYPE MEM_TYPE IS ARRAY (0 TO 31) OF STD_LOGIC_VECTOR(31 DOWNTO 0);

    SIGNAL myarray : MEM_TYPE := (x"00000000", 
        x"00000001", 
        x"00000002", 
        x"00000003", 
        x"00000004", 
        x"00000005", 
        x"00000006", 
        x"00000007", 
        x"00000008", 
        x"00000009", 
        x"00000010", 
        x"00000011", 
        x"00000012", 
        x"00000013", 
        x"00000014", 
        x"00000015", 
        x"00000016", 
        x"00000017", 
        x"00000018", 
        x"00000019", 
        x"00000020", 
        x"00000021", 
        x"00000022", 
        x"00000023", 
        x"00000024", 
        x"00000025", 
        x"00000026", 
        x"00000027", 
        x"00000028", 
        x"00000029", 
        x"00000030", 
    x"00000031"); 
BEGIN
    PROCESS (clock, MemWrite, MemRead)
    BEGIN
        IF falling_edge(clock) THEN 
            IF (MemWrite = '1') THEN
                myarray(TO_INTEGER(UNSIGNED(Address)/4)) <= DataIn; 
            ELSIF (MemRead = '1') THEN
                DataOut <= myarray(TO_INTEGER(UNSIGNED(Address)/4));
            END IF;
        END IF;
    END PROCESS;
END internal;
\end{lstlisting}

\clearpage

\begin{lstlisting}[language=VHDL,label=lst-mux32,caption={32-bit 2-to-1 Multiplexer (Mux\_2\_1.vhd)}]
LIBRARY IEEE;
USE IEEE.STD_LOGIC_1164.ALL;
USE IEEE.STD_LOGIC_UNSIGNED.ALL;

ENTITY Mux_2_1 IS
    PORT (
        A, B : IN STD_LOGIC_VECTOR(31 DOWNTO 0);
        Sel : IN STD_LOGIC;
    R : OUT STD_LOGIC_VECTOR(31 DOWNTO 0)); 
END Mux_2_1;

ARCHITECTURE behavioral OF Mux_2_1 IS
BEGIN
    PROCESS (A, B, Sel)
    BEGIN
        IF Sel = '0' THEN
            R <= A;
        ELSIF Sel = '1' THEN
            R <= B;
        END IF;
    END PROCESS;
END behavioral;
\end{lstlisting}

\clearpage

\begin{lstlisting}[language=VHDL,label=lst-mux5,caption={5-bit 2-to-1 Multiplexer (Mux\_2\_1\_5b.vhd)}]
LIBRARY IEEE;
USE IEEE.STD_LOGIC_1164.ALL;
USE IEEE.STD_LOGIC_UNSIGNED.ALL;

ENTITY Mux_2_1_5b IS
    PORT (
        A, B : IN STD_LOGIC_VECTOR(4 DOWNTO 0);
        Sel : IN STD_LOGIC;
        R : OUT STD_LOGIC_VECTOR(4 DOWNTO 0)
    );
END Mux_2_1_5b;

ARCHITECTURE behavioral OF Mux_2_1_5b IS
BEGIN
    PROCESS (A, B, Sel)
    BEGIN
        IF Sel = '0' THEN
            R <= A;
        ELSIF Sel = '1' THEN
            R <= B;
        END IF;
    END PROCESS;
END behavioral;
\end{lstlisting}

\clearpage

\begin{lstlisting}[language=VHDL,label=lst-shiftl2,caption={Shift Left 2 Unit (ShiftL2.vhd)}]
LIBRARY IEEE;
USE IEEE.STD_LOGIC_1164.ALL;
USE IEEE.STD_LOGIC_UNSIGNED.ALL;

ENTITY ShiftL2 IS
    PORT (
        A : IN STD_LOGIC_VECTOR(31 DOWNTO 0);
        Y : OUT STD_LOGIC_VECTOR(31 DOWNTO 0)
    );
END ShiftL2;

ARCHITECTURE internal OF ShiftL2 IS
BEGIN
    Y <= A(29 DOWNTO 0) & "00";
END internal;
\end{lstlisting}




\end{document}
