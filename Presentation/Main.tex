\documentclass[t,aspectratio=169,xcolor=dvipsnames]{beamer}
\usefonttheme{professionalfonts}
\usetheme{SimplePlusAIC}
\usepackage{graphicx} % Allows including images
\usepackage{booktabs} % Allows the use of \toprule, \midrule and  \bottomrule in tables
\usepackage{svg} % Allows using svg figures
\usepackage{tikz}
\usepackage{makecell}
\newcommand*{\defeq}{\stackrel{\text{def}}{=}}
\usepackage{setspace}
\usepackage[T1]{fontenc}
\usepackage{helvet}
\usepackage{textgreek}
\usepackage{amsmath}
\usepackage{ragged2e}
\usepackage{listings}
\usepackage{xfrac}
\usepackage[nice]{nicefrac}
\usepackage[loose]{units}
\usepackage{braket}
\usepackage{physics}
\usepackage{textcomp} % Replacing gensymb for symbols like \micro and \perthousand
\usepackage[cm]{sfmath}
\usepackage{bm}
\usepackage{verbatim}
\usepackage{fancyvrb}
\usepackage{hyperref}
\usepackage[style=ieee]{citation-style-language}
\addbibresource{references.bib}
\setbeamercovered{transparent}
\beamerdefaultoverlayspecification{<+->}
\setbeamertemplate{blocks}[rounded][shadow] % block options
\usepackage[svgnames,table]{xcolor}
\usepackage{xcolor}

\definecolor{codegreen}{rgb}{0,0.6,0}
\definecolor{codegray}{rgb}{0.5,0.5,0.5}
\definecolor{codepurple}{rgb}{0.58,0,0.82}
\definecolor{backcolour}{rgb}{0.95,0.95,0.92}

\lstdefinestyle{mystyle}{
    backgroundcolor=\color{backcolour},   
    commentstyle=\color{codegreen},
    keywordstyle=\color{magenta},
    numberstyle=\tiny\color{codegray},
    stringstyle=\color{codepurple},
    basicstyle=\ttfamily\footnotesize,
    breakatwhitespace=false,         
    breaklines=true,                 
    captionpos=b,                    
    keepspaces=true,                 
    numbers=left,                    
    numbersep=5pt,                  
    showspaces=false,                
    showstringspaces=false,
    showtabs=false,                  
    tabsize=2
}

\lstset{style=mystyle}

\arrayrulecolor{black}
\setlength{\arrayrulewidth}{0.20mm}
\renewcommand{\arraystretch}{1.40}  % stretch tables row size
\tracinglostchars=0

%================================================
%	TITLE PAGE
%================================================
\title[short title]{EENG 5342 Project} 
\subtitle{Single-Cycle MIPS Processor}

\author[Surname]{\texorpdfstring{Anish Goyal and Killian McCrary}{Anish Goyal and Killian McCrary}}
\institute[CBI]{Department of Electrical/Computer Engineering \\ Georgia Southern University \\ Statesboro, GA \\ }

\date{\textcolor{nyublue}{December 3, 2025}}

%================================================
%	BEGIN DOCUMENT 
%================================================
\begin{document}
\nocite{*}

%------------------------------------------------
%	TITLE SLIDE
%------------------------------------------------
\settitlepagebackground
\begin{frame}[plain]
    \titlepage
\end{frame}
\resetbackground

%------------------------------------------------
%	OUTLINE SLIDE
%------------------------------------------------
{
\beamerdefaultoverlayspecification{}
\begin{frame}[plain]
    \frametitle{Outline}
    \begin{columns}[t]
        \begin{column}{.5\textwidth}
            \begin{spacing}{1.3}
                \tableofcontents[sections={1-3}]
            \end{spacing}
        \end{column}
        \begin{column}{.5\textwidth}
            \begin{spacing}{1.2}
                \tableofcontents[sections={4-6}]
            \end{spacing}
        \end{column}
    \end{columns}
\end{frame}
}

\setsectioncontent{An overview of the task at hand and what we planned to accomplish.}
\section{Introduction}

\subsection{Motivation}
\begin{frame}
    \frametitle{Motivation}
    \framesubtitle{Why we chose this project}
    
    \begin{itemize}
        \item Gain hands-on experience with processor architecture
        \item Understand the datapath and control unit interaction
        \item Apply VHDL skills to a complex, multi-component system
        \item Bridge the gap between theoretical CPU design and practical implementation
    \end{itemize}
\end{frame}

\subsection{Abstract}
\begin{frame}
    \frametitle{Abstract} 
    \framesubtitle{Overview of the project and its goals}
    
    \begin{itemize}
        \item Implemented a single-cycle MIPS processor in VHDL
        \item Supports R-type, I-type, and memory instructions
        \item Key components include:
        \begin{itemize}
            \item Program Counter (PC)
            \item Instruction Memory
            \item Register File (32 registers)
            \item ALU with controller
            \item Data Memory
            \item Main Control Unit
        \end{itemize}
        \item Validated via functional simulation in Questa/ModelSim
    \end{itemize}
\end{frame}

\subsection{What was planned}
\begin{frame}
    \frametitle{What was planned}
    \framesubtitle{Our objectives}
    
    \begin{itemize}
        \item Create a fully functional single-cycle MIPS datapath
        \item Support the following instruction types:
        \begin{itemize}
            \item R-type: \texttt{add}, \texttt{sub}, \texttt{and}, \texttt{or}, \texttt{slt}
            \item I-type: \texttt{lw}, \texttt{sw}
        \end{itemize}
        \item Integrate all components via schematic capture in Quartus
        \item Verify correctness through waveform simulation
    \end{itemize}
\end{frame}

\setsectioncontent{Detailed explanation of the approach and techniques used.}
\section{Methodology}

\subsection{Design}
\begin{frame}
    \frametitle{Design}
    \framesubtitle{Overview of the design process}
    
    \begin{itemize}
        \item Based on the classic MIPS single-cycle architecture
        \item Each instruction executes in one clock cycle
        \item Datapath connects:
        \begin{itemize}
            \item PC $\rightarrow$ Instruction Memory $\rightarrow$ Register File
            \item ALU performs computation
            \item Data Memory for load/store operations
        \end{itemize}
        \item Control signals generated by Main Control and ALU Controller
    \end{itemize}
\end{frame}

\setsectioncontent{Details on how the processor was built and tested.}
\section{Implementation}

\subsection{Block Diagram}
\begin{frame}
    \frametitle{Block Diagram}
    \framesubtitle{Schematic overview of the datapath}
    
    \begin{itemize}
        \item Full datapath designed using Quartus Block Diagram File (.bdf)
        \item Components instantiated as VHDL symbols
        \item Wiring completed manually in schematic editor
        \item Video walkthrough of the block schematic:
    \end{itemize}
    
    \vspace{0.5cm}
    \centering
    \href{https://youtu.be/19uMkeUoLpE}{\texttt{\underline{https://youtu.be/19uMkeUoLpE}}}
\end{frame}

\subsection{Opcode Implementation}
\begin{frame}
    \frametitle{Opcode Implementation}
    \framesubtitle{Control signal generation based on instruction opcode}
    
    \begin{columns}
        \begin{column}{0.45\textwidth}
            \begin{itemize}
                \item Main Control decodes 6-bit opcode
                \item Generates control signals:
                \begin{itemize}
                    \item \texttt{RegDst}, \texttt{ALUSrc}
                    \item \texttt{MemtoReg}, \texttt{RegWrite}
                    \item \texttt{MemRead}, \texttt{MemWrite}
                    \item \texttt{Branch}, \texttt{ALUOp}
                \end{itemize}
                \item ALU Controller uses \texttt{ALUOp} and \texttt{funct} field
            \end{itemize}
        \end{column}
        \begin{column}{0.55\textwidth}
            \begin{figure}
                \centering
                \includegraphics[width=\textwidth]{figures/Opcode Table.png}
                \caption{Opcode table for control signal generation}
            \end{figure}
        \end{column}
    \end{columns}
\end{frame}

\subsection{Key Components}
\begin{frame}
    \frametitle{Key Components}
    \framesubtitle{Two important components of the datapath}
    
    \begin{columns}
        \begin{column}{0.5\textwidth}
            \textbf{ALU Controller}
            \begin{itemize}
                \item Receives \texttt{ALUOp} from Main Control
                \item Decodes \texttt{funct} field for R-type
                \item Outputs 4-bit ALU control signal
                \item Supports: ADD, SUB, AND, OR, SLT
            \end{itemize}
        \end{column}
        \begin{column}{0.5\textwidth}
            \textbf{Register File}
            \begin{itemize}
                \item 32 general-purpose registers
                \item Dual read ports, single write port
                \item Synchronous write on clock edge
                \item Register \$0 hardwired to zero
            \end{itemize}
        \end{column}
    \end{columns}
\end{frame}

\setsectioncontent{Summary of simulation results.}
\section{Results}

\subsection{Sample Waveform}
\begin{frame}[fragile]
    \frametitle{Sample Waveform}
    \framesubtitle{Functional simulation output}
    
    \textbf{Initial Conditions:}
    \begin{itemize}
        \item Registers \texttt{\$0} -- \texttt{\$31} initialized to values \texttt{0} -- \texttt{31}
    \end{itemize}
    
    \vspace{0.3cm}
    \textbf{Test Instructions:}
    \begin{Verbatim}[fontsize=\small]
x"012a4020"  -- 0x012A4020 -> add $8, $9, $10
x"01285022"  -- 0x01285022 -> sub $10, $9, $8
x"8e510000"  -- 0x8E510000 -> lw $17, 0($18)
x"ae290004"  -- 0xAE290004 -> sw $9, 4($17)
    \end{Verbatim}
\end{frame}

\begin{frame}
    \frametitle{Sample Waveform}
    \framesubtitle{Simulation results}
    
    \begin{figure}
        \centering
        \includegraphics[width=0.95\textwidth]{figures/Waveform.png}
        \caption{Waveform output from Questa/ModelSim functional simulation}
    \end{figure}
\end{frame}

\setsectioncontent{Challenges encountered and potential future work.}
\section{Challenges and Improvements}

\subsection{Problems During Design}
\begin{frame}
    \frametitle{Problems During Design}
    \framesubtitle{Issues encountered during implementation}
    
    \begin{itemize}
        \item \textbf{Bus naming convention} -- inconsistent naming caused wiring issues
        \item \textbf{Quartus schematic editor} -- unreliable click registration and net changes
        \item \textbf{Input/output pin naming} -- required careful matching
        \item \textbf{Constant value generation} -- needed explicit constant blocks
        \item \textbf{Functional simulation} -- required \texttt{voptargs="+acc"} on personal laptop
    \end{itemize}
\end{frame}

\begin{frame}[fragile]
    \frametitle{Problems During Design}
    \framesubtitle{Additional issues}
    
    \begin{itemize}
        \item \textbf{XOR overriding SUB} -- class code had XOR opcode conflicting with SUB; required control value changes
        \item \textbf{Questa license server} -- needed to set \texttt{SALT\_LICENSE\_SERVER} environment variable
        \item \textbf{MemToReg mux polarity} -- flipped polarity inherent in reference diagram
        \item \textbf{5-bit 2-to-1 mux} -- needed to create a special 6-bit mux before register file
    \end{itemize}
\end{frame}

\subsection{Potential Improvements}
\begin{frame}
    \frametitle{Potential Improvements}
    \framesubtitle{Future work and extensions}
    
    \begin{itemize}
        \item \textbf{Half-period asymmetry} -- add half cycle of zero before first rising edge
        \item \textbf{Pipelining} -- implement 5-stage pipeline with data hazard detection
        \item \textbf{Jump support} -- add unconditional jump (\texttt{j}) instruction
        \item \textbf{Real-time output} -- display register/memory values to console
        \item \textbf{Live instruction input} -- avoid recompilation for instruction changes
    \end{itemize}
\end{frame}

\begin{frame}
    \frametitle{Potential Improvements}
    \framesubtitle{Additional enhancements}
    
    \begin{itemize}
        \item \textbf{Pipeline registers} -- add registers between stages
        \item \textbf{Additional operations} -- multiplication, division, shifting
        \item \textbf{Data forwarding} -- bypass paths for hazard mitigation
        \item \textbf{FPGA deployment} -- synthesize and run on actual hardware
    \end{itemize}
\end{frame}

\subsection{Conclusion}
\begin{frame}
    \frametitle{Conclusion}
    \framesubtitle{Summary of results}
    
    \begin{itemize}
        \item Successfully implemented a single-cycle MIPS processor
        \item Supports R-type (\texttt{add}, \texttt{sub}) and I-type (\texttt{lw}, \texttt{sw}) instructions
        \item Verified functionality through waveform simulation
        \item Gained practical experience with datapath design and control logic
        \item Identified areas for future improvement and extension
    \end{itemize}
\end{frame}    

\beamerdefaultoverlayspecification{}
\begin{frame}[plain]
    \frametitle{References}
    \printbibliography
\end{frame}

\conclusionpage
\end{document}
